% mn2e-example.tex
%
% This is a skeleton MN-format article which displays the generated
% test bibliography (mn2e-test.bbl) as it would appear in an article.
% To see this, try:
%
%     make mn2e-example.pdf
%
% Note that the mn2e-test.aux file is written by hand; the
% mn2e-example.aux file generated from this file is ignored.

\documentclass[authoryear,10pt,5p,times]{elsarticle}

\journal{Astronomy and Computing}

\begin{document}

\begin{frontmatter}
\title{Illustrations of Citations}

\author[ao]{Albert One}
\ead{ao@example.ac.uk}
\address[ao]{A Place, Somewhere}

\author[bt]{Benedict Two}
\address[bt]{Elsewhere, On Earth}

\begin{abstract}
It is important to make reference to things.
\end{abstract}

\end{frontmatter}

% The following implements the three-author-hack described in
% mn2e.bst.  This should be moved to mn2e.cls at some point.
%
% This consumes a command for each such author.  It's surely possible
% to avoid this (with some constructions involving {\\#1}; see
% Appendix D cleverness), but that would verge on the arcane, and not
% be really worth it.
%% \makeatletter
%% \def\mniiiauthor#1#2#3{%
%%   \@ifundefined{mniiiauth@#1}
%%     {\global\expandafter\let\csname mniiiauth@#1\endcsname\null #2}
%%     {#3}}
%% \makeatother

\section{Introduction}

There are numerous citations here.

In particular, there is
`one'~\citep{one},
`oneplus'~\citep{oneplus},
`onereprised'~\citep{onereprised},
`two'~\citep{two},
`twoplus'~\citep{twoplus},
`twobis'~\citep{twobis},
`twobook'~\citep{twobook},
`twomisc~\citep{twomisc},
`tworeprised'~\citep{tworeprised},
`three'~\citep{three},
`threeplus'~\citep{threeplus},
`threereprised'~\citep{threereprised},
`four'~\citep{four},
`fourplus'~\citep{fourplus},
`seven'~\citep{seven},
`sevenplus'~\citep{sevenplus},
`eight'~\citep{eight},
`eightplus'~\citep{eightplus},
`nine'~\citep{nine},
`nineplus'~\citep{nineplus},
`ten'~\citep{ten},
`tenplus'~\citep{tenplus} and
`tenbis'~\citep{tenbis}.

And secondly, there is
`one'~\citep{one},
`oneplus'~\citep{oneplus},
`onereprised'~\citep{onereprised},
`two'~\citep{two},
`twoplus'~\citep{twoplus},
`twobis'~\citep{twobis},
`twobook'~\citep{twobook},
`twomisc~\citep{twomisc},
`tworeprised'~\citep{tworeprised},
`three'~\citep{three},
`threeplus'~\citep{threeplus},
`threereprised'~\citep{threereprised},
`four'~\citep{four},
`fourplus'~\citep{fourplus},
`seven'~\citep{seven},
`sevenplus'~\citep{sevenplus},
`eight'~\citep{eight},
`eightplus'~\citep{eightplus},
`nine'~\citep{nine},
`nineplus'~\citep{nineplus},
`ten'~\citep{ten},
`tenplus'~\citep{tenplus} and
`tenbis'~\citep{tenbis}.



% ../model2-names.bst can generate a \logsortkey in the output, for
% debugging.  That's generally commented out, but if you need to
% uncomment that, then uncomment this, too.
%\def\logsortkey#1{{[\tiny #1]}}
%\input{model2-names-test.bbl}

\bibliographystyle{model2-names-astronomy}
\bibliography{model2-names-test}

\end{document}
